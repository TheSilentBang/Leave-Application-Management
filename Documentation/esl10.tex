\documentclass[11pt]{article}
\usepackage{graphicx}
\setlength\parindent{0pt}
\usepackage{listings}
\usepackage{times}
\usepackage{color}


\title{Assignment 10}
\date{ 28/09/2014}
\begin{document}
\maketitle

\section*{Problem Statement}
Demonstrate following skills to solve difficulties / challenges in current project.
Posting a question on the blog or forum, 
writing and maintaining a mailing list mails, 
Writing outputs or logs using PostBin tools or equivalent, 
use of GIT(refer to github Web site) for revision control or open source equivalent.
\section*{Objective}\
\begin{itemize}
\item To learn use of advance programming, documentation, presentation and communication tools.
\item To learn use to group discussions in problem solving.
\item To learn quantitative skills
\end{itemize}

\section*{Outcomes}\
\begin{itemize}
\item Ability to understand need of technical competence required for problem solving
\item Ability to understand the flaws in the current product and make changes according to the review.
\end{itemize}

\section*{Theory}
Difficulties and challenges faced:
To create an account for every student and teacher SQL is required. Functions to be used need to show details of the account holder from database and also store the applications filed by the student in the database. \\
The Application entered by the students needs to be saved in the server database. As currently the project has been executed on the localhost machines, therefore all the data needs to be accessed using php5. At first we counldnt find a way to connect out database using php on the machine. Then we found a post on StackOverFlow.com where we found several posts which helped us tackle through our dificulties. We also used GitHUB to  create a repository of our project where we found many pulls requests where other developers helped us to make our code better and any further bugs which occurred during their execution.
\section*{Conclusion}
Thus,we have taken into consideration the review given by the customer and will be making changes accordingly.
\end{document}